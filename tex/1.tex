\section{目次}
 \begin{itemize}
  \item 対称行列
 \end{itemize}

 \section{対称行列}
 wikiを地道に解読します。\\
 線形代数学における対称行列は、自身の転置行列と一致するような正方行列を言う。
 記号で書けば、行列Aは
 \begin{eqnarray}
  A = A^T 
 \end{eqnarray}
 を満たすときの対象であるという。相等しい行列の型は相等しいから、この式を満すのは正方行列に限られる。

 定義により、対称行列の成分は主対角線に関して対象である。すなわち、成分に関して行列$A=(a_{ij})$は任意の添字に関して$a_{ij} = a_{ji}$を満たす。例えば、次の$3 \times 3$行列

 \begin{eqnarray}
  \left(
   \begin{matrix}
	1&7&3\\
	7&4&-5\\
	3&-5&6
   \end{matrix}
   \right)
 \end{eqnarray}
は対象である。任意の正方対角行列は、その非対称対角成分が0であるから、対象である。


\subsection{対角行列って?}
非対角成分が0である行列。\\

同様に、歪対称行列(わいたいしょうぎょうれつ)の対角成分は、自身の転置行列が自身の-1倍となるものをいう。
 \begin{eqnarray}
  A^T = -A 
 \end{eqnarray}
